\documentclass[a4paper,11pt,bibliography=totoc,listof=totoc,headinclude=true,cleardoublepage=empty,oneside]{NumPDEsThesis}
% Option "oneside" für einseitigen Druck. Weglassen, falls die Arbeit doppelseitig gedruckt wird

% Stelle die Sprache ein: "ngerman" für Deutsch, "english" für Englisch
\selectlanguage{ngerman}

% Lade Literaturdatenbank
\addbibresource{literature.bib}

% Lade weitere Pakete
\usepackage{NumPDEsMacros} % in NumPDEsMacros.sty sind einige nützliche Befehle definiert

% Subfile-Paket für modulare Struktur
\usepackage{subfiles}

% Bevor die Arbeit gedruckt wird, sollten die Farben ausgeschaltet werden, indem die folgende Zeile auskommentiert wird.
% \hypersetup{colorlinks=false,citecolor=black,linkcolor=black,urlcolor=black,pagebackref=false}

\begin{document}

%%%%%%%%%%%%%%%%%%%%%%%%%%%%%%%%%%%%%%%%%%%%%%%%%%%%%%%%%%%%%%%%%%%%%%%%%%%%%%%%%%%%%%%%%%%%%%%%%%%%%%%%%%%%%%
% TITELSEITE [OBLIGATORISCH]
%%%%%%%%%%%%%%%%%%%%%%%%%%%%%%%%%%%%%%%%%%%%%%%%%%%%%%%%%%%%%%%%%%%%%%%%%%%%%%%%%%%%%%%%%%%%%%%%%%%%%%%%%%%%%%

\pagenumbering{Alph}

\begin{titlepage}
  %\vspace*{-2cm}
  \begin{center}
    \includegraphics[width=0.45\textwidth]{figures/TULogo.pdf}
    \vskip 1cm%
    {\LARGE B~\Large A~C~H~E~L~O~R~A~R~B~E~I~T}
    \vskip 8mm
    {\huge\bfseries Perfekte Graphen und Algorithmen \\[1ex] für bestimmte Teilklassen davon}
    \vskip 1cm
    \large 
    ausgef\"uhrt am    
    \vskip 0.75cm
    {\Large Institut f\"ur\\[1ex] Diskrete Mathematik und Geometrie}\\[1ex]
    {\Large TU Wien}
    \vskip0.75cm
    unter der Anleitung von
    \vskip0.75cm
    {\Large\bfseries Ao.Univ.Prof. Bernhard Gittenberger}\\[1ex]
    \vskip 0.5cm
    durch
    \vskip 0.5cm
    {\Large\bfseries Florian Bogner}\\[1ex]
    Matrikelnummer: {01225415}\\[1ex]
    {Absberggasse 5/21}\\[1ex]
    {1100 Wien}
  \end{center}
  
  \vfill
  
  \small
  Wien, am {\today} %\today
  \vspace*{-15mm}
\end{titlepage}

\cleardoublepage

%%%%%%%%%%%%%%%%%%%%%%%%%%%%%%%%%%%%%%%%%%%%%%%%%%%%%%%%%%%%%%%%%%%%%%%%%%%%%%%%%%%%%%%%%%%%%%%%%%%%%%%%%%%%%%
% DANKSAGUNG / ACKNOWLEDGEMENT [OPTIONAL]
%%%%%%%%%%%%%%%%%%%%%%%%%%%%%%%%%%%%%%%%%%%%%%%%%%%%%%%%%%%%%%%%%%%%%%%%%%%%%%%%%%%%%%%%%%%%%%%%%%%%%%%%%%%%%%

\chapter*{Danksagung} %\chapter*{Acknowledgement}
\thispagestyle{empty}

\begin{itemize}
\item auf Deutsch oder Englisch
\item Die Danksagung (engl. {\em Acknowledgement}) ist optional und kann auch entfallen. Denken Sie ggf.\ an Ihre eigenen Eltern!

\item Falls die Arbeit durch eine Forschungsprojekt finanziert wurde, so ist jedenfalls der Fördergeber (z.B.\ FWF oder WWTF) mit Projektnummer und Projektname zu nennen.
\begin{itemize}
\item siehe z.B.\ Dissertation von Michele Ruggeri:
\item[] \href{https://publik.tuwien.ac.at/files/publik_252806.pdf}{\ttfamily https://publik.tuwien.ac.at/files/publik\_252806.pdf}
\end{itemize}

\end{itemize}

\cleardoublepage

%%%%%%%%%%%%%%%%%%%%%%%%%%%%%%%%%%%%%%%%%%%%%%%%%%%%%%%%%%%%%%%%%%%%%%%%%%%%%%%%%%%%%%%%%%%%%%%%%%%%%%%%%%%%%%
% EIDESSTATTLICHE ERKLAERUNG [OBLIGATORISCH]
%%%%%%%%%%%%%%%%%%%%%%%%%%%%%%%%%%%%%%%%%%%%%%%%%%%%%%%%%%%%%%%%%%%%%%%%%%%%%%%%%%%%%%%%%%%%%%%%%%%%%%%%%%%%%%

\chapter*{Eidesstattliche Erkl\"arung}
\thispagestyle{empty}

\vspace*{2cm}

Ich erkl\"are an Eides statt, dass ich die vorliegende Bachelorarbeit selbstst\"andig und ohne fremde Hilfe verfasst, andere als die angegebenen Quellen und Hilfsmittel nicht benutzt bzw. die w\"ortlich oder sinngem\"a{\ss} entnommenen Stellen als solche kenntlich gemacht habe.

\vspace*{3cm}

\noindent
Wien, am {\color{change}Datum} %\today
%
\hfill 
%
\begin{minipage}[t]{5cm}
\centering
\underline{\hspace*{5cm}}\\
\small\color{change}Name des Autors
\end{minipage}

\cleardoublepage

%%%%%%%%%%%%%%%%%%%%%%%%%%%%%%%%%%%%%%%%%%%%%%%%%%%%%%%%%%%%%%%%%%%%%%%%%%%%%%%%%%%%%%%%%%%%%%%%%%%%%%%%%%%%%%
% INHALTSVERZEICHNIS [OBLIGATORISCH]
%%%%%%%%%%%%%%%%%%%%%%%%%%%%%%%%%%%%%%%%%%%%%%%%%%%%%%%%%%%%%%%%%%%%%%%%%%%%%%%%%%%%%%%%%%%%%%%%%%%%%%%%%%%%%%

\pagenumbering{roman}

\tableofcontents

\cleardoublepage
\pagenumbering{arabic} 

%%%%%%%%%%%%%%%%%%%%%%%%%%%%%%%%%%%%%%%%%%%%%%%%%%%%%%%%%%%%%%%%%%%%%%%%%%%%%%%%%%%%%%%%%%%%%%%%%%%%%%%%%%%%%%
% KAPITEL DER ARBEIT
%%%%%%%%%%%%%%%%%%%%%%%%%%%%%%%%%%%%%%%%%%%%%%%%%%%%%%%%%%%%%%%%%%%%%%%%%%%%%%%%%%%%%%%%%%%%%%%%%%%%%%%%%%%%%%

% Verwende subfiles für modulare Struktur der Arbeit
% Es bietet sich an, jedes Kapitel in einem eigenen File zu schreiben und mit \subfile{filename} einzubinden
% Die subfiles können auch einzeln kompiliert werden, indem sie als Hauptdatei gesetzt werden

%\subfile{chapters/einleitung}
%\subfile{chapters/manual}
\subfile{chapters/triangulierte_graphen.tex}

%%%%%%%%%%%%%%%%%%%%%%%%%%%%%%%%%%%%%%%%%%%%%%%%%%%%%%%%%%%%%%%%%%%%%%%%%%%%%%%%%%%%%%%%%%%%%%%%%%%%%%%%%%%%%%
% LITERATURVERZEICHNIS MIT BIBLATEX --> MATHSCINET VERWENDEN
%%%%%%%%%%%%%%%%%%%%%%%%%%%%%%%%%%%%%%%%%%%%%%%%%%%%%%%%%%%%%%%%%%%%%%%%%%%%%%%%%%%%%%%%%%%%%%%%%%%%%%%%%%%%%%

\printbibliography

\end{document}
